%%%%%%%%%%%%%%%%%%%%%%%%%%%%%%%%%%%%%%%%%
% University/School Laboratory Report
% LaTeX Template
% Version 3.1 (25/3/14)
%
% This template has been downloaded from:
% http://www.LaTeXTemplates.com
%
% Original author:
% Linux and Unix Users Group at Virginia Tech Wiki 
% (https://vtluug.org/wiki/Example_LaTeX_chem_lab_report)
%
% License:
% CC BY-NC-SA 3.0 (http://creativecommons.org/licenses/by-nc-sa/3.0/)
%
%%%%%%%%%%%%%%%%%%%%%%%%%%%%%%%%%%%%%%%%%

%----------------------------------------------------------------------------------------
%	PACKAGES AND DOCUMENT CONFIGURATIONS
%----------------------------------------------------------------------------------------

\documentclass{article}

\usepackage{graphicx} % Required for the inclusion of images
\usepackage{amsmath} % Required for some math elements 
\usepackage{enumitem}
\usepackage{hyperref}
\usepackage[left=25mm,right=25mm,top=25mm,bottom=25mm]{geometry}
\usepackage{makecell}
\usepackage{float}

\title{Report: \textbf{University Rankings}} % Title

\author{Julian \textsc{Backé} \\ Tobias \textsc{Salzer} \\ Ajayvir \textsc{Singh}} % Author name


\begin{document}

\maketitle % Insert the title, author and date

In this work, we answer the following questions:

\begin{itemize}
	\item How do university rankings change over time? 
	
	\item Which characteristics of universities contribute most to good rankings, or to large changes in the ranking position? 
	
	\item How do these characteristics correlate with characteristics of cities or countries in which the university is located? 
	
	\item Are there predictors for increases or decreases in the rankings?
\end{itemize}

\section*{\large{How do university rankings change over time?}}
We obtained that the majority of universities have a rather stable ranking. This means that for most universities there are no huge changes in the rankings. However, there are also some (few) universities that jump in the rankings.

\section*{\large{Which characteristics of universities contribute most to good rankings, or to large changes in the ranking position?}}
todo

\section*{\large{How do these characteristics correlate with characteristics of cities or countries in which the university is located?}}

The results of this section are summed up in Figure 1. For example, we obtained that a country's expenditure on education (in percent of the GDP) leads to a higher mean score of this country's universities. However, a country's expenditures is not indicator on how good its best university will rank. We also found out that the number of universities per inhabitant as well as the Human Development Index (HDI) could be indicators for how good a country's universities will perform on average in rankings. It should be noted that with these characteristics it is possible to distinguish the mean ranks of different countries, but it is not possible to separate good and bad universities within the very same country

\begin{figure}[H]
\caption{Which local properties influence a country's university scores?}
\begin{center}
\begin{tabular}{|c|c|c|} \hline
\textbf{independent variable} & \textbf{dependent variable} & \textbf{impact} \\ \hline
\makecell{expenditures for education \\ (all institutions)} & mean score of country & YES \\ \hline
\makecell{expenditures for education \\ (all institutions)} & max. score of country & NO \\ \hline
\makecell{expenditures for education \\ (higher institutions)} & mean score of country & YES \\ \hline
\makecell{expenditures for education \\ (higher institutions)} & max. score of country & NO \\ \hline
\makecell{number of universities} & mean score of country & SLIGHT \\ \hline
\makecell{number of inhabitants} & mean score of country & NO \\ \hline
\makecell{univerisites per inhabitant} & mean score of country & YES \\ \hline
\makecell{HDI} & mean score of country & YES \\ \hline
\makecell{corruption} & mean score of country & todo \\ \hline
\end{tabular}
\end{center}
\end{figure}


\section*{\large{Are there predictors for increases or decreases in the rankings?}}
We constructed two models predicting a university's score. Once we considered only the number of students, the student staff ratio, the percentage of female students and the percentage of international students. For this model a so called Random Forest algorithm was used. In our tests, our model an average error of 3.38 score points, which is a rather good model. When using additioal properties of the university's location country like the HDI or the data about expenditure for education, our model worked slightly worse. This makes sense because within a country, the universities might have very different scores. However, these additional numbers are the same for each university in the same country, so that they are not suitable for distinguishing different universities in the same country.





\end{document}